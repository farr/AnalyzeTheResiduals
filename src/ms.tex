% Define document class
\documentclass[modern]{aastex631}

% Filler text
\usepackage{blindtext}

% Math shortcuts
\newcommand{\like}{\mathcal{L}}
\newcommand{\order}[1]{\mathcal{O}\left( #1 \right)}

% Begin!
\begin{document}

% Title
\title{When and How Can we Fit Residuals?}

% Author list
\author[0000-0003-1540-8562]{Will M. Farr}
\email{will.farr@stonybrook.edu}
\email{wfarr@flatironinstitute.org}
\affiliation{Department of Physics and Astronomy, Stony Brook University, Stony Brook NY 11794, USA}
\affiliation{Center for Computational Astrophysics, Flatiron Institute, New York NY 10010, USA}

% Abstract with filler text
\begin{abstract}
    I discuss the effect of fixing the \emph{residuals} from a global fit in
    LISA-like data when fitting broadband signals like the inspiral of a
    high-redshift seed BBH.
\end{abstract}

% Main body with filler text
\section{Introduction}

The joint likelihood for a single BBH merger waveform $h$ and some number of
white dwarf monochromatic or near-monochromatic waveforms in LISA data $d$ is
%
\begin{equation}
    \log \like \propto - \Delta f \sum_i \frac{\left| d_i - h_i - \sum_\alpha g_{\alpha, i} \right|^2}{2 S_i},
\end{equation}
%
where the sum runs over frequency-domain components and $S_i$ is the noise PSD
at frequency $i$.  Expanding to quadratic order about $h_i = h_{i,0}$ in terms
of the parameters $\theta$ that control the waveform, we have
%
\begin{multline}
    \log \like \sim - \Delta f \sum_i \frac{1}{2 S_i} \left( \left| d_i - h_{i,0} - G_i \right|^2 - 2 \Re \left( d_i - h_{i,0} - G_i \right)^* \frac{\partial h_i}{\partial \theta^a} \left( \theta^a - \theta^a_0 \right)  \right. \\ \left. + \left(\theta^a - \theta_0^a \right) \left( \frac{\partial h_i^*}{\partial \theta^a} \frac{\partial h_i}{\partial \theta^b} + \Re \left( d_i - h_{i,0} - G_i \right)^* \frac{\partial^2 h_i}{\partial \theta^a \partial \theta^b} \right) \left( \theta^b - \theta_0^b \right) \right),
\end{multline}
%
where
%
\begin{equation}
    G_i \equiv \sum_\alpha g_{\alpha,i}.
\end{equation}
%
Collecting terms, we see that we have a Gaussian likelihood for the parameters
$\theta$, with
%
\begin{equation}
    \log \like \sim - \left( A + B_a \Delta \theta^a + \frac{1}{2} \Delta \theta^a C_{ab} \Delta \theta^b \right),
\end{equation}
%
with
%
\begin{equation}
    B_a = \Delta f \sum_i \frac{1}{S_i} \left( - \Re \left( d_i - h_{i,0} - G_i \right)^* \frac{\partial h_i}{\partial \theta^a} \right)
\end{equation}
%
and
%
\begin{equation}
    C_{ab} = \Delta f \sum_i \frac{1}{S_i} \left( \frac{\partial h_i^*}{\partial \theta^a} \frac{\partial h_i}{\partial \theta^b} + \Re \left( d_i - h_{i,0} - G_i \right)^* \frac{\partial^2 h_i}{\partial \theta^a \partial \theta^b} \right).
\end{equation}
%

Now we make the assumption that the sources making up $G$ are \emph{narrowband},
that is that their S/N accumulates over a small number of bins around $i = i_0$,
such that
%
\begin{equation}
\frac{\left| g_{\alpha,i} \right|^2}{S_i} \sim \begin{cases}
    \order{\rho_\alpha^2} & \left| i - i_0 \right| \sim \order{1} \\
    0 & \mathrm{otherwise}
\end{cases}.
\end{equation}
%
We further assume that $h$ and therefore $h_0$ are \emph{broadband}, so
%
\begin{equation}
\frac{\left| h_i \right|^2}{S_i} \sim \frac{\order{\rho_h^2}}{N}
\end{equation}
%
where $N \gg 1$ is the number of bins over which the S/N of $h$ accumulates.

Let us suppose that the data are composed of signals like $h$ and $G$ plus colored noise $n$ with PSD $S$:
%
\begin{equation}
    d_i = \bar{h}_i + \bar{G}_i + n_i
\end{equation}
%



\end{document}
