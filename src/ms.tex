% Define document class
\documentclass[modern]{aastex631}

% Filler text
\usepackage{blindtext}

% Math shortcuts
\newcommand{\like}{\mathcal{L}}

% Begin!
\begin{document}

% Title
\title{When and How Can we Fit Residuals?}

% Author list
\author[0000-0003-1540-8562]{Will M. Farr}
\email{will.farr@stonybrook.edu}
\email{wfarr@flatironinstitute.org}
\affiliation{Department of Physics and Astronomy, Stony Brook University, Stony Brook NY 11794, USA}
\affiliation{Center for Computational Astrophysics, Flatiron Institute, New York NY 10010, USA}

% Abstract with filler text
\begin{abstract}
    I discuss the effect of fixing the \emph{residuals} from a global fit in
    LISA-like data when fitting broadband signals like the inspiral of a
    high-redshift seed BBH.
\end{abstract}

% Main body with filler text
\section{Introduction}

The joint likelihood for a single BBH merger waveform $h$ and some number of
white dwarf monochromatic or near-monochromatic waveforms in LISA data $d$ is
%
\begin{equation}
    \log \like \propto - \sum_i \Delta f \frac{\left| d_i - h_i - \sum_\alpha g_{\alpha, i} \right|^2}{2 S_i},
\end{equation}
%
where the sum runs over frequency-domain components and $S_i$ is the noise PSD
at frequency $i$.


\end{document}
